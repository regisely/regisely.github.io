%% LyX 2.3.4.3 created this file.  For more info, see http://www.lyx.org/.
%% Do not edit unless you really know what you are doing.
\documentclass[10pt,english,brazil]{beamer}
\usepackage[T1]{fontenc}
\usepackage[utf8]{inputenc}
\setcounter{secnumdepth}{3}
\setcounter{tocdepth}{3}
\usepackage{amsthm}
\usepackage{graphicx}
\usepackage[numbers]{natbib}

\makeatletter

%%%%%%%%%%%%%%%%%%%%%%%%%%%%%% LyX specific LaTeX commands.
%% Because html converters don't know tabularnewline
\providecommand{\tabularnewline}{\\}

%%%%%%%%%%%%%%%%%%%%%%%%%%%%%% Textclass specific LaTeX commands.
% this default might be overridden by plain title style
\newcommand\makebeamertitle{\frame{\maketitle}}%
% (ERT) argument for the TOC
\AtBeginDocument{%
  \let\origtableofcontents=\tableofcontents
  \def\tableofcontents{\@ifnextchar[{\origtableofcontents}{\gobbletableofcontents}}
  \def\gobbletableofcontents#1{\origtableofcontents}
}

%%%%%%%%%%%%%%%%%%%%%%%%%%%%%% User specified LaTeX commands.
\usetheme{Warsaw}
\newtheorem{thm}{Teorema}[]
\newtheorem{cor}[]{Corollary}
\newtheorem{lem}[]{Lema}
\newtheorem{prop}[]{Proposi\c{c}\~ao}
\theoremstyle{definition}
\newtheorem{defn}[]{Definition}
\theoremstyle{remark}
\newtheorem{rem}[thm]{Remark}
\usepackage{amsthm}\usepackage{amsfonts}

\makeatother

\usepackage{babel}
\begin{document}
\title{\selectlanguage{english}%
Métodos Estatísticos Básicos}
\subtitle{Aula 4 - Medidas de dispersão}
\author{\selectlanguage{english}%
Prof. Regis Augusto Ely}
\institute{\selectlanguage{english}%
Departamento de Economia\\
Universidade Federal de Pelotas (UFPel)}
\date{\selectlanguage{english}%
Abril de 2014}
\makebeamertitle
\selectlanguage{brazil}%
\begin{frame}{Amplitude total}

\begin{itemize}
\item \textbf{Amplitude total:} $AT=X_{max}-X_{min}$. É a única medida
de dispersão que não tem na média o ponto de referência.
\item Para dados agrupados sem intervalos de classe, a fórmula é a mesma
acima. 
\item Para dados com intervalos de classe, $AT=L_{max}-l_{min}$, onde $l_{min}$
é o menor limite inferior das classes e \foreignlanguage{english}{\textrm{$L_{max}$}}
o maior limite superior.
\selectlanguage{english}%
\item Obs: a amplitude total desconsidera valores intermediários.
\end{itemize}
\end{frame}

\begin{frame}{Desvio quartil}

\begin{itemize}
\item \textbf{Desvio quartil:} $D_{q}=\frac{(Q3-Q1)}{2}.$ É também chamado
de amplitude semi-interquartílica.
\item Usamos o desvio quartil preferencialmente quando a medida de tendência
central utilizada é a mediana. 
\item O desvio quartil não é tao afetado por valores extremos.

Ex: \{40, 45, 48, 62, 70\}

$Q1=\frac{40+45}{2}=42,5$ e $Q3=\frac{62+70}{2}=66$

$D_{q}=\frac{66-42,5}{2}=11,75$
\end{itemize}
\end{frame}

\begin{frame}{Desvio médio absoluto}

\begin{itemize}
\item \textbf{Desvio médio absoluto:}$D_{m}=\frac{\overset{n}{\underset{i=1}{\sum}}|Xi-\bar{X}|}{n};$
$D_{me}=\frac{\overset{n}{\underset{i=1}{\sum}}|Xi-Me|}{n}$. É a
média aritmética dos valores absolutos dos desvios tomados em relação
à média ou à mediana.

Ex: \{-4, -3, -2, 3, 5\}

$\bar{X}=-0,2$ e $M_{e}=2$.

$D_{m}=\frac{|-4+0,2|+|-3+0,2|+|-2+0,2|+|3+0,2|+|5+0,2|}{5}=3,36$

$D_{me}=\frac{|-4+2|+|-3+2|+|-2+2|+|3+2|+|5+2|}{5}=3$
\item Para dados agrupados devemos utilizar as frequências, $D_{m}=\frac{\overset{n}{\underset{i=1}{\sum}}fi.(Xi-\bar{X})}{\overset{n}{\underset{i=1}{\sum}}fi},$
e se tivermos intervalos de classe, então Xi será o ponto médio de
cada classe.
\end{itemize}
\end{frame}

\begin{frame}{Diferença média}

\begin{itemize}
\item \textbf{Diferença média:} $\triangle=\frac{1}{n^{2}}.\overset{n}{\underset{i=1}{\sum}}\overset{n}{\underset{j=1}{\sum}}|Xi-Xj|$.
É o desvio absoluto em relação a todos os dados entre si.
\item Essa expressão pode ser simplificada para $\triangle=\frac{4}{n^{2}}.\overset{n}{\underset{i=1}{\sum}}i.Xi-2\bar{X}(1+\frac{1}{n})$
(ver pag. 51 de \textit{Hoffman}, \textit{R. Estatística para economistas}).
\end{itemize}
\end{frame}

\begin{frame}{Desvio-padrão}

\begin{itemize}
\item \textbf{Desvio padrão:} $\sigma=\sqrt{\frac{\overset{n}{\underset{i=1}{\sum}}(Xi-\overline{X})^{2}}{n}}$.
É a raíz da média aritmética dos quadrados dos desvios.

Ex: \{-4, -3, -2, 3, 5\}

$\bar{X}=-0,2$ 

$\sigma=\sqrt{\frac{(-4+0,2)^{2}+(-3+0,2)^{2}+(-2+0,2)^{2}+(3+0,2)^{2}+(5+0,2)^{2}}{5}}=\sqrt{12,56}=3,54$.
\item \textbf{Desvio padrão amostral:} $S=\sqrt{\frac{\overset{n}{\underset{i=1}{\sum}}(Xi-\bar{X})^{2}}{n-1}}$.
Utilizamos essa pequena correção no caso de termos apenas uma amostra
da população completa.
\end{itemize}
\end{frame}

\begin{frame}{Desvio-padrão}


\framesubtitle{Dados agrupados}
\begin{itemize}
\item Quando temos dados agrupados, devemos ponderar o desvio padrão pelas
frequências: 

$\sigma=\sqrt{\frac{\overset{n}{\underset{i=1}{\sum}}[(Xi-\overline{X})^{2}.fi]}{\overset{n}{\underset{i=1}{\sum}}fi}}$
quando se trata da população inteira.

$S=\sqrt{\frac{\overset{n}{\underset{i=1}{\sum}}[(Xi-\overline{X})^{2}.fi]}{\overset{n}{\underset{i=1}{\sum}}fi-1}}$
quando se trata de uma amostra.
\item Com intervalos de classe, Xi será o ponto médio da classe.
\end{itemize}
\end{frame}

\begin{frame}{Desvio-padrão e variância}


\framesubtitle{Propriedades}
\begin{itemize}
\item As principais propriedades do desvio padrão são:
\end{itemize}
\begin{enumerate}
\item Somando (ou subtraindo) uma constante a todos os valores de uma variável,
o desvio-padrão não se altera.
\item Multiplicando (ou dividindo) todos os valores de uma variável por
uma constante (diferente de zero), o desvio-padrão será multiplicado
(ou dividido) por essa constante.
\end{enumerate}
\begin{itemize}
\item \textbf{Variância }(\textbf{$\sigma^{2}$}ou $S^{2}$) \textbf{:}
é o desvio-padrão elevado ao quadrado. A propriedade 1 continua válida
para a variância, mas a propriedade 2 se altera, pois se multiplicarmos
todos os valores por uma constante (diferente de zero), a variância
será multiplicada por essa mesma constante elevada ao quadrado.
\end{itemize}
\end{frame}

\begin{frame}{Medidas de dispersão relativa}


\framesubtitle{Coeficiente de variação de Pearson (CVP)}
\begin{itemize}
\item \textbf{Coeficiente de variação de Pearson (CVP): }$CVP=\frac{\sigma}{\bar{X}}\times100$.
Caracteriza a dispersão dos dados em relação ao seu valor médio.
\item Um desvio padrão de 2 pode ser grande para dados cuja média é 20,
mas pequeno se a média é 200. O CVP padroniza as variações, possibilitando
a comparação entre dados distintos.
\item Ex:
\end{itemize}
\begin{center}
\begin{tabular}{|c|c|c|}
\hline 
variável & média & desvio\tabularnewline
\hline 
\hline 
altura & 175cm & 5,0cm\tabularnewline
\hline 
peso & 68kg & 2,0kg\tabularnewline
\hline 
\end{tabular} 
\par\end{center}

\begin{center}
Qual série é mais homogênea?
\par\end{center}

\begin{center}
=> $CVP_{altura}=\frac{5,0}{175}x100=2,85\%$. <=
\par\end{center}

\begin{center}
$CVP_{peso}=\frac{2,0}{68}x100=2,94\%$.
\par\end{center}
\begin{itemize}
\item \textbf{Coeficiente de variação de Thorndike (CVT):} $CVT=\frac{S}{Me}x100$.
Utilizamos a mediana para o cálculo. 
\end{itemize}
\end{frame}

\begin{frame}{Medidas de assimetria}

\begin{itemize}
\item \textbf{Distribuição simétrica: }dizemos que os dados tem uma distribuição
simétrica quando Média = Mediana = Moda.
\item \textbf{Distribuição assimétrica à esquerda: }é a assimétrica negativa,
que ocorre quando Média < Mediana < Moda.
\item \textbf{Distribuição assimétrica à direita: }é a assimétrica positiva,
que ocorre quando Média > Mediana > Moda.
\item \textbf{Coeficiente de assimetria de Pearson:} $CAP_{Me}=\frac{3.(\bar{X}-Me)}{\sigma}$
e $CAP_{Mo}=\frac{\bar{X}-Mo}{\sigma}.$ Compara graus de assimetria
entre distribuições diferentes. 
\item \textbf{Classificação:}

\begin{enumerate}
\item $|CAP|<0,15$ $\Rightarrow$ Assimetria pequena.
\item $0,15<|CAP|<1\Rightarrow$ Assimetria moderada.
\item $|CAP|>1\Rightarrow$ Assimetria elevada.
\end{enumerate}
\end{itemize}
\end{frame}

\begin{frame}{Medidas de assimetria}

\begin{itemize}
\item Se $CAP=0$, os dados tem distribuição simétrica. Se $CAP<0$ a assimetria
é negativa. Se $CAP>0$ a assimetria é positiva.
\end{itemize}
\begin{center}
\includegraphics[scale=0.4]{assimetria}
\par\end{center}

\end{frame}

\begin{frame}{Medidas de curtose}

\begin{itemize}
\item \textbf{Curtose: }é o grau de achatamento de uma distribuição em relação
à distribuição normal (em forma de sino).
\item \textbf{Distribuição leptocúrtica:} apresenta uma distribuição mais
alongada do que a normal.
\item \textbf{Distribuição platicúrtica:} apresenta uma distribuição mais
achatada do que a normal.
\item \textbf{Distribuição mesocúrtica:} distribuição não é nem achatada
nem alongada (igual a da normal).
\end{itemize}
\end{frame}

\begin{frame}{Medidas de curtose}

\begin{itemize}
\item \textbf{Percentílico de curtose:} $C1=\frac{(Q3-Q1)}{2(P90-P10)}$.

$C1=0,263\Rightarrow$ curva mesocúrtica.

$C1<0,263\Rightarrow$ curva leptocúrtica.

$C1>0,263\Rightarrow$ curva platicúrtica.
\item \textbf{Momento de curtose:} $K=\frac{\underset{i=1}{\sum}(Xi-\overline{X})^{4}.fi}{\overset{n}{\underset{i=1}{\sum}}fi}\frac{1}{S^{4}}$.
\item $K=3$ \foreignlanguage{english}{\textrm{$\Rightarrow$}} curva mesocúrtica.
\item $K>3$ $\Rightarrow$ curva leptocúrtica.
\item $K<3$ $\Rightarrow$ curva platicúrtica.
\end{itemize}
\end{frame}

\begin{frame}{Medidas de curtose}

\begin{itemize}
\item Os valores dos coeficientes de curtose determinam o grau de achatamento
da distribuição. Graficamente,
\end{itemize}
\begin{center}
\includegraphics[scale=0.35]{preview006}
\par\end{center}
\end{frame}

\end{document}
