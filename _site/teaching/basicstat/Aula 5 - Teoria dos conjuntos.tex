%% LyX 2.3.4.3 created this file.  For more info, see http://www.lyx.org/.
%% Do not edit unless you really know what you are doing.
\documentclass[10pt,english,brazil]{beamer}
\usepackage[T1]{fontenc}
\usepackage[utf8]{inputenc}
\setcounter{secnumdepth}{3}
\setcounter{tocdepth}{3}
\usepackage{slashed}
\usepackage{amstext}
\usepackage{amsthm}
\usepackage{amssymb}
\usepackage{graphicx}
\usepackage[numbers]{natbib}

\makeatletter
%%%%%%%%%%%%%%%%%%%%%%%%%%%%%% Textclass specific LaTeX commands.
% this default might be overridden by plain title style
\newcommand\makebeamertitle{\frame{\maketitle}}%
% (ERT) argument for the TOC
\AtBeginDocument{%
  \let\origtableofcontents=\tableofcontents
  \def\tableofcontents{\@ifnextchar[{\origtableofcontents}{\gobbletableofcontents}}
  \def\gobbletableofcontents#1{\origtableofcontents}
}

%%%%%%%%%%%%%%%%%%%%%%%%%%%%%% User specified LaTeX commands.
\usetheme{Warsaw}
\newtheorem{thm}{Teorema}[]
\newtheorem{cor}[]{Corollary}
\newtheorem{lem}[]{Lema}
\newtheorem{prop}[]{Proposi\c{c}\~ao}
\theoremstyle{definition}
\newtheorem{defn}[]{Definition}
\theoremstyle{remark}
\newtheorem{rem}[thm]{Remark}
\usepackage{amsthm}\usepackage{amsfonts}

\makeatother

\usepackage{babel}
\begin{document}
\title{\selectlanguage{english}%
Métodos Estatísticos Básicos}
\subtitle{Aula 5 - Teoria dos conjuntos}
\author{\selectlanguage{english}%
Prof. Regis Augusto Ely}
\institute{\selectlanguage{english}%
Departamento de Economia\\
Universidade Federal de Pelotas (UFPel)}
\date{\selectlanguage{english}%
Abril de 2014}
\makebeamertitle
\selectlanguage{brazil}%
\begin{frame}{Definições básicas}

\selectlanguage{english}%
\begin{itemize}
\item \textbf{Conjunto:} é uma coleção de objetos, sendo representados por
letras maiúsculas, A, B, etc.

Ex: A=\{1, 2, 3, 4, 5, 6\} => Resultados possíveis de um lançamento
de dado.

Ex: B=\{$x/0\leq x\leq1$\} => Composto pelo intervalo {[}0,1{]}.
\item \textbf{Elementos: }são os objetos que compõem o conjunto.

Ex: $3\in A$, $0,5\in B$,$2\notin B$.
\item \textbf{Conjunto vazio: }é o conjunto $\textrm{Ø}$, que não contém
nenhum elemento.
\item \textbf{Subconjuntos: }se, dados dois conjuntos A e B, $x\in A$ =>
$x\in B$, então dizemos que A é um subconjunto de B, $A\subset B$.
Se também vale que $B\subset A$, então os dois conjuntos são iguais,
$A=B$.
\end{itemize}
\end{frame}

\begin{frame}{Definições básicas}

\selectlanguage{english}%
\begin{itemize}
\item Dizemos que o conjunto vazio é subconjunto de qualquer outro conjunto,
ou seja $\textrm{Ø}\subset A$, $\textrm{Ø}\subset B$.
\item \textbf{Conjunto fundamental}: é o conjunto $U$ de todos os objetos
que estão sendo estudados, de forma que $A,$$B\subset U$.
\item Note que para qualquer conjunto A, temos $\oslash\subset A\subset U$.

Ex: $U=\mathbb{R}$, $A=\{x|x^{2}+2x-3=0\}$, $B=\{x|(x-2)(x^{2}+2x-3)=0\}$,
$C=\{-3,1,2\}.$

Prove que $A\subset B\subset U$ e $B=C$.
\end{itemize}
\end{frame}
\selectlanguage{brazil}%

\begin{frame}{Definições básicas}

\selectlanguage{english}%
\begin{itemize}
\item Há duas operações fundamentais sobre os conjuntos, a \textbf{união}
e a \textbf{interseção}.
\item C é a união de A e B se $C=\{x|x\in A$ e/ou $x\in B\}$. Assim, $C=A\cup B$.
\item D é a interseção de A e B se $D=\{x|x\in A$ e $x\in B\}$. Assim,
$D=A\cap B$.
\item \textbf{Conjunto complemento: }é o conjunto $\bar{A}$ que contém
todos os elementos que não estão em A, mas estão no conjunto fundamental
$U$. Assim, $\bar{A}=\{x\in U|x\notin A\}$.
\end{itemize}
\end{frame}

\begin{frame}{\foreignlanguage{english}{Propriedades das operações de conjuntos}}

\selectlanguage{english}%
\begin{itemize}
\item \textbf{Leis comutativas:}

a) $A\cup B=B\cup A$.

b) $A\cap B=B\cap A$.
\item \textbf{Leis associativas:}

c) $A\cup(B\cup C)=(A\cup B)\cup C$. 

d)$A\cap(B\cap C)=(A\cap B)\cap C$.
\end{itemize}
\end{frame}
\selectlanguage{brazil}%

\begin{frame}{\foreignlanguage{english}{Propriedades das operações de conjuntos}}

\selectlanguage{english}%
\begin{itemize}
\item \textbf{Outras propriedades: }

e)$A\cup(B\cap C)=(A\cup B)\cap(A\cup C)$. 

f) $A\cap(B\cup C)=(A\cap B)\cup(A\cap C)$. 

g) $A\cap\slashed{O}=\slashed{O}$.

h) $A\cup\slashed{O}=A$.

i) $\overline{(A\cup B)}=\bar{A}\cap\bar{B}$.

j) $\overline{A\cap B}=\overline{A}\cup\bar{B}$.

k) $\overline{\overline{A}}=A$.
\end{itemize}
\end{frame}
\selectlanguage{brazil}%

\begin{frame}{\foreignlanguage{english}{Diagramas de Venn}}

\selectlanguage{english}%
\begin{itemize}
\item \textbf{Diagrama de Venn}: representação gráfica das operações de
conjuntos, sendo a região sombreada o conjunto sob exame.
\item $A\cup B$ 

\includegraphics[scale=0.15]{1-1}
\item $A\cap B$ 

\includegraphics[scale=0.15]{2-1}
\item $\overline{A}$

\includegraphics[viewport=0bp 0bp 400bp 185bp,scale=0.2]{A-}
\selectlanguage{brazil}%
\item \textit{Exercício: }demonstre as propriedades e, f, j, i e k com Diagramas
de Venn.
\end{itemize}
\end{frame}

\begin{frame}{\foreignlanguage{english}{Produto cartesiano}}

\selectlanguage{english}%
\begin{itemize}
\item \textbf{Produto cartesiano: }o produto cartesiano de A e B, é o conjunto
$A\times B=\{(a,b)|a\in A,b\in B\}$, ou seja, é composto pelos pares
ordenados nos quais o primeiro elemento é tirado de A e o segundo
de B.

Ex: $A=\{1,2,3\}$; $B=\{1,2,3,4\}$; $AxB=\{(1,1),(1,2),...,(1,4),(2,1),...,(2,4),(3,1),...,(3,4)\}$.
\item Note que $A\times B\neq B\times A$.
\item A noção de produto cartesiano pode ser estendida, de modo que $A_{1}\times A_{2}\times...\times A_{n}=\{(a_{1},a_{2},...,a_{n})|a_{i}\in A_{i}\}.$
\end{itemize}
\end{frame}

\begin{frame}{\foreignlanguage{english}{Conjuntos enumeráveis}}

\selectlanguage{english}%
\begin{itemize}
\item O número de elementos que um conjunto possui será de grande importância
para nós, pois veremos que em probabilidade, os possíveis resultados
de um experimento são expressos como elementos de um conjunto.
\item \textbf{Conjunto finito: }um conjunto A é finito se existir um número
finito de elementos nele. Ex:$A=\{a_{1},a_{2},...,a_{n}\}$.
\item \textbf{Conjunto enumerável: }um conjunto B é enumerável, ou infinito
enumerável, se existir uma correspondência biunívoca entre os elementos
de B e os números inteiros positivos. Ex: $\mathbb{N=}\{1,2,3,...\}$,
$B=\{2,4,6,8,...\}.$
\item \textbf{Conjunto infinito não-enumerável: }um conjunto C é infinito
não-enumerável se possuir um número infinito de elementos que não
podem ser enumerados. Ex: $C=\{x|a\leq x\leq b\}$ para quaisquer
números reais $b>a$. Note que qualquer intervalo (não-degenerado)
contém mais do que um número contável de pontos.\selectlanguage{brazil}%
\end{itemize}
\end{frame}

\end{document}
