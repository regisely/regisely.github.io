%% LyX 2.3.4.3 created this file.  For more info, see http://www.lyx.org/.
%% Do not edit unless you really know what you are doing.
\documentclass[10pt,english,brazil]{beamer}
\usepackage[T1]{fontenc}
\usepackage[utf8]{inputenc}
\setcounter{secnumdepth}{3}
\setcounter{tocdepth}{3}
\usepackage{fancybox}
\usepackage{calc}
\usepackage{units}
\usepackage{textcomp}
\usepackage{amstext}
\usepackage{amsthm}
\usepackage{amssymb}
\usepackage[numbers]{natbib}
\PassOptionsToPackage{normalem}{ulem}
\usepackage{ulem}

\makeatletter
%%%%%%%%%%%%%%%%%%%%%%%%%%%%%% Textclass specific LaTeX commands.
% this default might be overridden by plain title style
\newcommand\makebeamertitle{\frame{\maketitle}}%
% (ERT) argument for the TOC
\AtBeginDocument{%
  \let\origtableofcontents=\tableofcontents
  \def\tableofcontents{\@ifnextchar[{\origtableofcontents}{\gobbletableofcontents}}
  \def\gobbletableofcontents#1{\origtableofcontents}
}
\theoremstyle{definition}
\newtheorem*{example*}{\protect\examplename}
\theoremstyle{plain}
\newtheorem*{thm*}{\protect\theoremname}

%%%%%%%%%%%%%%%%%%%%%%%%%%%%%% User specified LaTeX commands.
\usetheme{Warsaw}
\newtheorem{thm}{Teorema}[]
\newtheorem{cor}[]{Corollary}
\newtheorem{lem}[]{Lema}
\newtheorem{prop}[]{Proposi\c{c}\~ao}
\theoremstyle{definition}
\newtheorem{defn}[]{Definition}
\theoremstyle{remark}
\newtheorem{rem}[thm]{Remark}
\usepackage{amsthm}\usepackage{amsfonts}

\makeatother

\usepackage{babel}
\addto\captionsbrazil{\renewcommand{\examplename}{Exemplo}}
\addto\captionsbrazil{\renewcommand{\theoremname}{Teorema}}
\addto\captionsenglish{\renewcommand{\examplename}{Example}}
\addto\captionsenglish{\renewcommand{\theoremname}{Theorem}}
\providecommand{\examplename}{Exemplo}
\providecommand{\theoremname}{Teorema}

\begin{document}
\title{\selectlanguage{english}%
Métodos Estatísticos Básicos}
\subtitle{Aula 6 - Introdução à probabilidade}
\author{\selectlanguage{english}%
Prof. Regis Augusto Ely}
\institute{\selectlanguage{english}%
Departamento de Economia\\
Universidade Federal de Pelotas (UFPel)}
\date{\selectlanguage{english}%
Maio de 2014}
\makebeamertitle
\selectlanguage{brazil}%
\begin{frame}{Experimento}

\selectlanguage{english}%
\begin{itemize}
\item \textbf{Experimento aleatório ($E$):} é um experimento que pode ser
repetido indefinidamente sob condições essencialmente inalteradas.
\item Embora não possamos descrever um resultado particular do experimento,
podemos descrever o conjunto de todos os possíveis resultados e as
probabilidades associadas a eles. Isso porque repetindo o experimento
um grande número de vezes, uma regularidade surgirá.
\item A descrição de um experimento envolve um procedimento a ser realizado
e uma observação a ser constatada.

Ex 1: Jogue um dado e observe o número mostrado na face de cima.

Ex 2: Jogue uma moeda 4 vezes e observe o número de caras obtido.

Ex 3: Receba duas cartas de um baralho e observe quantos ases foram
obtidos.

Ex 4: Um míssil é lançado. Em momentos específicos $t_{1}t_{2},...,t_{n},$
a altura do míssil acima do solo é registrada.
\end{itemize}
\end{frame}
\selectlanguage{brazil}%

\begin{frame}{\foreignlanguage{english}{Espaço Amostral}}

\selectlanguage{english}%
\begin{itemize}
\item \textbf{Espaço amostral ($\Omega$): }é o conjunto de todos os resultados
possíveis de um experimento $E$. 
\item Note a semelhança do espaço amostral com o conjunto fundamental $U$.
Um espaço amostral está sempre associado a um experimento e este conjunto
nem sempre é composto de números.

Ex 1: $\Omega=\{1,2,3,4,5,6\}$.

Ex 2: $\Omega=\{0,1,2,3,4\}.$

Ex 3: $\Omega=\{0,1,2\}.$

Ex 4: $\Omega=\{h_{1},h_{2},...,h_{n}|h_{i}\geq0,i=1,2,...,n\}.$

Ex 5: Jogue uma moeda 2 vezes e obtenha a sequência de caras e coroas
obtidas. $\Omega=\{(H,H),(H,T),(T,H),(T,T)\}.$
\selectlanguage{brazil}%
\item O número de elementos de um espaço amostral pode ser finito, infinito
enumerável ou infinito não-enumerável. Todo resultado possível de
um experimento corresponde a um, e somente um ponto $w\in\Omega$,
sendo que resultados distintos correspondem a pontos distintos.
\end{itemize}
\end{frame}

\begin{frame}{\foreignlanguage{english}{Eventos}}

\selectlanguage{english}%
\begin{itemize}
\item \textbf{Evento: }um evento $A\subset\Omega$ é um conjunto de resultados
possíveis do experimento $E$, mas não necessariamente todos.

Ex 1: Um número par ocorre, $A=\{2,4,6\}.$

Ex 2: Duas caras ocorrem, $A=\{2\}$.

Ex 3: Obtemos apenas um Ás, $A=\{1\}$.

Ex 5: Obtemos pelo menos uma cara, $A=\{(H,H),(H,T),(T,H)\}$.
\item Qualquer um desses eventos é um subconjunto de $\Omega$. O evento
$\Omega$ é chamado \textit{evento certo}; o evento $\textrm{Ø}$
é chamado \textit{evento impossível}, e o evento \{$\omega$\} é dito
\textit{elementar}.
\end{itemize}
\end{frame}

\begin{frame}{\foreignlanguage{english}{Operações com eventos}}

\selectlanguage{english}%
\begin{itemize}
\item \textbf{Eventos compostos: }$A\cup B$ é o evento ``A ou B'', $A\cap B$
é o evento ``A e B'' e $\overline{A}$ é o evento ``não A''.
\item Se $A_{1},...,A_{n}$ for qualquer coleção finita de eventos, então
$\cup_{i=1}^{n}A_{i}$ será o evento que ocorrerá se, e somente se,
ao menos um dos eventos $A_{i}$ocorrer. Já $\cap_{i=1}^{n}A_{i}$
será o evento que ocorrerá se, e somente se, todos os eventos $A_{i}$
ocorrerem.
\item Os mesmos resultados se estendem para coleções infinitas enumeráveis
$A_{1},A_{2},...,A_{n},...$, sendo $\cup_{i=1}^{\infty}A_{i}$ e
$\cap_{i=1}^{\infty}A_{i}$ os respectivos conjuntos.
\item \textbf{Dependência: }$A\subset B$ significa que a ocorrência do
evento A implica a ocorrência do evento B.
\item \textbf{Eventos mutuamente excludentes: }$A\cap B=\textrm{Ø}$ significa
que A e B são eventos que nunca ocorrem juntos, ou disjuntos.
\end{itemize}
\end{frame}

\begin{frame}{Características dos eventos}

\selectlanguage{english}%
\begin{itemize}
\item \textbf{Produto cartesiano de eventos: }se executarmos um experimento
$E$ duas vezes, então nosso espaço amostral será $\Omega x\Omega$
e os eventos AxA. Isso pode ser estendido para $n$ vezes.
\item \textbf{Proposição}: se o espaço amostral $\Omega$ for finito, com
$n$ elementos, entao existirá exatamente $2^{n}$ subconjuntos de
$\Omega$, ou seja, eventos.

Ex: lançar uma moeda e verificar o resultado. $\Omega=\{H,T\}$. Número
de subconjuntos é $2^{2}=4$, sendo eles $\{\textrm{Ø\},\{H\},\{T\},\{H,T\}}.$
\item \textit{Dica: }para enumerar subconjuntos de um espaço amostral basta
pensar em termos matemáticos quais são os subconjuntos de $\Omega$.
Não pensem em termos dos resultados do experimento.
\end{itemize}
\end{frame}

\begin{frame}{\foreignlanguage{english}{Definição clássica de probabilidade}}

\begin{itemize}
\item A \textit{\uline{definição clássica de probabilidade}} aplica-se
apenas se o espaço amostral é finito e os resultados do experimento,
$\omega\in\Omega$, são igualmente verossímeis:

\shadowbox{\begin{minipage}[t]{0.8\paperwidth}%
$P(A)=\cfrac{n\text{º de resultados favoráveis à \ensuremath{A}}}{n\text{º de resultados possíveis}}=\cfrac{n\text{º de elementos de \ensuremath{A}}}{n\text{º de elementos de \ensuremath{\Omega}}}$.%
\end{minipage}}.
\item Esta definição de probabilidade é aplicável apenas a um número restrito
de problemas (que envolvem escolha ao acaso e resultados finitos do
experimento).

\begin{example*}
$E=$ jogar um dado e observar o número de cima $\Rightarrow$ $\Omega=\{1,2,3,4,5,6\}$;

$A=$ obter um número par $\Rightarrow$ $A=\{2,4,6\}$ ;

$P(A)=\frac{3}{6}=\frac{1}{2}=50\%$.
\end{example*}
\end{itemize}
\end{frame}

\begin{frame}{\foreignlanguage{english}{Definição frequentista de probabilidade}}

\begin{itemize}
\item Repetindo n vezes um experimento $E$, podemos definir a frequência
relativa do evento A como $f_{A}=\frac{n_{A}}{n}$. Se repetirmos
muitas vezes $E$, de modo que $n\rightarrow\infty$, teremos a \textit{\uline{definição
frequentista de probabilidade.}}

\shadowbox{\begin{minipage}[t]{0.3\columnwidth}%
$P(A)=\underset{n\rightarrow\infty}{\lim}\cfrac{n_{A}}{n}$. %
\end{minipage}}
\item Apenas podemos utilizar esta definição quando temos a possibilidade
de realizar o experimento muitas vezes, sendo o resultado suscetível
ao valor de n.
\end{itemize}
\end{frame}

\begin{frame}{Probabilidade geométrica}

\begin{itemize}
\item A \textit{\uline{definição de probabilidade geométrica}} nos diz
que dois eventos tem a mesma probabilidade se, e somente se, eles
têm a mesma área.

\shadowbox{\begin{minipage}[t]{0.35\columnwidth}%
$P(A)=\cfrac{\text{área de A}}{\text{ área de \ensuremath{\Omega}}}$.%
\end{minipage}}
\begin{example*}
$E=$ escolher um ponto ao acaso no círculo unitário;

$\Omega=\{(x,y)\in\mathbb{R}|x^{2}+y^{2}\leq1\}$;

$A=1\text{ª}$ coordenada do ponto escolhido é maior que a 2ª;

$A=\{(x,y)\in\Omega|x>y\}$;

$P(A)=\cfrac{\nicefrac{\pi}{2}}{\pi}=\cfrac{1}{2}$. (Lembre que a
área de um círculo é $A=\pi\times r^{2}$).
\end{example*}
\item Essa probabilidade é mais utilizada com espaços amostrais infinitos
não-enumeráveis.
\end{itemize}
\end{frame}

\begin{frame}{Definição matemática de probabilidade}

\begin{itemize}
\item A probabilidade de um evento $A\subset\Omega$ associado à $E$, e
denotada por P(A), é um número real que satisfaz as seguintes propriedades:
\end{itemize}
\begin{enumerate}
\item $0\leq P(A)\leq1$;
\item $P(\Omega)=1$;
\item Se $A\cap B=\textrm{Ø}$ (eventos mutuamente excludentes), então
$P(A\cup B)=P(A)+P(B)$;
\item Se $A_{1},A_{2},...,A_{n},...,$ forem, dois a dois, eventos mutuamente
excludentes, então $P(U_{i=1}^{\infty}A_{i})=P(A_{1})+...+P(A_{n})+...$
\end{enumerate}
\begin{itemize}
\item A propriedade 3 também vale para um número finito de uniões, $P(U_{i=1}^{n}A_{i})=\sum_{i=1}^{n}P(A_{i})$.
\item Da propriedade 3 decorre que $P(\textrm{Ø)=0},$ pois $P(A\cup\textrm{Ø})=P(A)+P(\textrm{Ø})=P(A)$.
\item Das propriedades 2 e 3 decorre que $P(\bar{A})=1-P(A)$, pois $P(A\cup\bar{A})=P(A)+P(\bar{A})=P(\Omega)=1.$
\end{itemize}
\end{frame}

\begin{frame}{Probabilidade da união de eventos}

\begin{thm*}
Se A e B forem dois eventos quaisquer, não necessariamente excludentes,
então:

$P(A\cup B)=P(A)+P(B)-P(A\cap B)$.
\end{thm*}
\begin{proof}
Note que $A\cup\bar{A}=\Omega$ e $A\cup B=(A\cup B)\cap(A\cup\bar{A})=A\cup(B\cap\bar{A}),$
sendo $A$ e $(B\cap\bar{A})$ eventos excludentes, de modo que $P(A\cup B)=P(A)+P(B\cap\bar{A})$
(1). Analogamente, $B=B\cap(A\cup\bar{A})=(B\cap A)\cup(B\cap\bar{A}),$
de modo que $P(B)=P(A\cap B)+P(B\cap\bar{A})$ (2). Juntando (1) e
(2), temos $P(A\cup B)=P(A)+P(B)-P(A\cap B)$.
\end{proof}
\end{frame}

\begin{frame}{Probabilidade da união de eventos}

\begin{thm*}
Se A, B e C forem 3 eventos quaisquer, então:

$P(A\cup B\cup C)=P(A)+P(B)+P(C)-P(A\cap B)-P(A\cap C)-P(B\cap C)+P(A\cap B\cap C)$.
\end{thm*}
\begin{proof}
Escreva $A\cup B\cup C$ na forma $(A\cup B)\cup C$ e aplique o resultado
do teorema anterior, de modo que $P((A\cup B)\cup C)=P(A\cup B)+P(C)-P((A\cup B)\cap C).$
Note que $P((A\cup B)\cap C)=P((A\cap C)\cup(B\cap C))=P(A\cap C)+P(B\cap C)-P(A\cap B\cap C)$,
e $P(A\cup B)=P(A)+P(B)-P(A\cap B)$. Logo, $P((A\cup B)\cup C)=P(A)+P(B)-P(A\cap B)+P(C)-P(A\cap C)-P(B\cap C)+P(A\cap B\cap C).$
\end{proof}
\begin{itemize}
\item O teorema acima pode ser estendido para n eventos.
\end{itemize}
\end{frame}

\begin{frame}{Probabilidade de subconjuntos}

\begin{thm*}
Se $A\subset B$, então $P(A)\leq P(B)$.
\end{thm*}
\begin{proof}
Como $A\cup B=B$, podemos decompor B em dois eventos mutuamente excludentes,
$B=A\cup(B\cap\bar{A})$, de modo que $P(B)=P(A)+P(B\cap\overline{A})\geq P(A)$.
\end{proof}
\begin{itemize}
\item Note que quaisquer das definições de probabilidade vistas anteriormente
devem respeitar as propriedades matemáticas que desenvolvemos.
\item Quando atribuímos uma probabilidade a um evento A chamamos ele de
\emph{\uline{evento aleatório}}\emph{.}
\end{itemize}
\end{frame}

\begin{frame}{Álgebra de eventos aleatórios}

\begin{itemize}
\item Uma \textit{\uline{álgebra de eventos }}\uline{aleatórios},
$\mathcal{A}$, é a coleção (classe) de subconjuntos de $\Omega$
que possui algumas propriedades básicas:

1. $\Omega\in\mathcal{A}$;

2. $B\in\mathcal{A}\Rightarrow\bar{B}\in\mathcal{A}$;

3. $A\in\mathcal{A}$ e $B\in\mathcal{A}$ $\Rightarrow$ $A\cup B\in\mathcal{A}.$
\item As seguintes propriedades decorrem de 1, 2 e 3:

4. $\textrm{Ø}\in\mathcal{A}$;

5. $\forall n$ e $\forall B_{1},B_{2},...B_{n}\in\mathcal{A}$, temos
$\overset{n}{\underset{i=1}{\cup}}B_{i}\in\mathcal{A}$ e $\overset{n}{\underset{i=1}{\cap}}B_{i}\in\mathcal{A}$
.
\item Dizemos que a álgebra é fechada para um número finito de aplicações
das operações $\cup$ e $\cap$.
\item Se estas propriedades forem válidas para um número infinito enumerável
de aplicações de $\cup$ e $\cap$, então chamamos $\mathcal{A}$
de $\sigma-\acute{a}lgebra$.
\end{itemize}
\end{frame}

\begin{frame}{Espaço de probabilidades}

\begin{itemize}
\item O nosso \textit{\emph{modelo probabilístico}}\emph{ }estará situado
dentro de um \textit{\uline{espaço de probabilidade}} $(\Omega,\mathcal{A},P)$,
constituído de:
\end{itemize}
\begin{enumerate}
\item Um conjunto não-vazio $\Omega$ de resultados possíveis do experimento
$E$, chamado espaço amostral;
\item Uma álgebra de eventos aleatórios $\mathcal{A}$, composta por todos
os subconjuntos de $\Omega$;
\item Uma probabilidade P definida sobre os conjuntos de $\mathcal{A}$,
com as propriedades matemáticas vistas anteriormente.
\end{enumerate}
\begin{itemize}
\item É nesse espaço que trabalharemos, sendo sempre importante identificar
$\Omega,\mathcal{A}$ e P.
\end{itemize}
\end{frame}

\end{document}
