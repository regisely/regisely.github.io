%% LyX 2.3.4.3 created this file.  For more info, see http://www.lyx.org/.
%% Do not edit unless you really know what you are doing.
\documentclass[10pt,english,brazil]{beamer}
\usepackage[T1]{fontenc}
\usepackage[utf8]{inputenc}
\setcounter{secnumdepth}{3}
\setcounter{tocdepth}{3}
\usepackage{textcomp}
\usepackage{amstext}
\usepackage{amsthm}
\usepackage[numbers]{natbib}
\PassOptionsToPackage{normalem}{ulem}
\usepackage{ulem}

\makeatletter

%%%%%%%%%%%%%%%%%%%%%%%%%%%%%% LyX specific LaTeX commands.
%% Because html converters don't know tabularnewline
\providecommand{\tabularnewline}{\\}

%%%%%%%%%%%%%%%%%%%%%%%%%%%%%% Textclass specific LaTeX commands.
% this default might be overridden by plain title style
\newcommand\makebeamertitle{\frame{\maketitle}}%
% (ERT) argument for the TOC
\AtBeginDocument{%
  \let\origtableofcontents=\tableofcontents
  \def\tableofcontents{\@ifnextchar[{\origtableofcontents}{\gobbletableofcontents}}
  \def\gobbletableofcontents#1{\origtableofcontents}
}

%%%%%%%%%%%%%%%%%%%%%%%%%%%%%% User specified LaTeX commands.
\usetheme{Warsaw}
\newtheorem{thm}{Teorema}[]
\newtheorem{cor}[]{Corollary}
\newtheorem{lem}[]{Lema}
\newtheorem{prop}[]{Proposi\c{c}\~ao}
\theoremstyle{definition}
\newtheorem{defn}[]{Definition}
\theoremstyle{remark}
\newtheorem{rem}[thm]{Remark}
\usepackage{amsthm}\usepackage{amsfonts}

\makeatother

\usepackage{babel}
\begin{document}
\title{\selectlanguage{english}%
Métodos Estatísticos Básicos}
\subtitle{Aula 3 - Medidas de tendência central}
\author{\selectlanguage{english}%
Prof. Regis Augusto Ely}
\institute{\selectlanguage{english}%
Departamento de Economia\\
Universidade Federal de Pelotas (UFPel)}
\date{\selectlanguage{english}%
Abril de 2014}
\makebeamertitle
\selectlanguage{brazil}%
\begin{frame}{Média aritmética}


\framesubtitle{Definição}
\begin{itemize}
\item As medidas de tendência central são estatísticas que caracterizam
um conjunto de dados, sendo o valor em torno do qual se agrupam as
observações.
\item \textbf{Média aritmética $(\overline{X})$:} é o quociente entre a
soma dos valores dos nossos dados e o número total de dados, $\bar{X}=\frac{{\sum_{i=1}^{n}X_{i}}}{n}$
.
\item Quando temos dados não-agrupados pela frequência das observações,
calculamos a \textit{\uline{média aritmética simples}}.

Ex: 10, 14, 13, 15, 16, 18, 12.

Logo, $\overline{X}=\frac{(10+14+13+15+16+18+12)}{7}=14$.
\item \textbf{Desvio em relação à média:} é a diferença entre um elemento
de um conjunto de valores e a média aritmética desse conjunto, ou
seja, $d_{i}=X_{i}-\overline{X}.$

Ex: $d_{1}=10-14=-4$; $d_{2}=14-14=0$, etc.
\end{itemize}
\end{frame}

\begin{frame}{Média aritmética}


\framesubtitle{Propriedades da média}
\selectlanguage{english}%
\begin{enumerate}
\item A soma algébrica dos desvios em relação à média é nula. Assim,$\overset{n}{\underset{i=1}{\sum}}d_{i}=\overset{n}{\underset{i=1}{\sum}}(Xi-\overline{X})=\overset{n}{\underset{i=1}{\sum}}Xi-n.\overline{X}=0$.
\item Somando ou subtraindo uma constante c a todos os elementos do nosso
conjunto de dados, a média aumentará em c. Assim, $\frac{\sum_{i=1}^{n}(Xi+c)}{n}=\frac{\sum_{i=1}^{n}Xi}{n}+\frac{n.c}{n}=\overline{X}+c$.
\item Multiplicando ou dividindo todos os valores por uma constante c, a
média fica multiplicada (ou dividida) por c. Assim, $\frac{\sum_{i=1}^{n}(c.Xi)}{n}=\frac{c\sum_{i=1}^{n}Xi}{n}=c.\overline{X}.$
\end{enumerate}
\end{frame}

\begin{frame}{Média aritmética ponderada}


\framesubtitle{Dados agrupados sem intervalo de classe}
\begin{itemize}
\item Se os dados estiverem agrupados em uma tabela de frequência, devemos
calcular uma \textit{\uline{média aritmética ponderada}}. $\overline{X}=\frac{\sum_{i=1}^{n}X_{i}f_{i}}{\sum_{i=1}^{n}fi}$.
Ex:
\end{itemize}
\begin{center}
\begin{tabular}{|c|c|}
\hline 
Dados & Frequência\tabularnewline
\hline 
\hline 
0 & 2\tabularnewline
\hline 
1 & 6\tabularnewline
\hline 
2 & 10\tabularnewline
\hline 
3 & 12\tabularnewline
\hline 
4 & 4\tabularnewline
\hline 
Total & 34\tabularnewline
\hline 
\end{tabular} 
\par\end{center}

\begin{center}
$\overline{X}=\frac{0x2+1x6+2x10+3x12+4x4}{34}=2,3$
\par\end{center}

\end{frame}

\begin{frame}{Média aritmética ponderada}


\framesubtitle{Dados agrupados com intervalos de classe}
\begin{itemize}
\item Se os dados estiverem agrupados em intervalos de classe, utilizamos
a \textit{\uline{média aritmética ponderada}}, definindo Xi como
o ponto médio da classe i. Ex:
\end{itemize}
\begin{center}
\begin{tabular}{|c|c|c|c|}
\hline 
Estaturas(cm) & Frequência (fi) & Ponto médio (Xi) & Xi.fi\tabularnewline
\hline 
\hline 
$50\vdash54$ & 4 & 52 & 208\tabularnewline
\hline 
$54\vdash58$ & 9 & 56 & 504\tabularnewline
\hline 
$58\vdash62$ & 11 & 60 & 660\tabularnewline
\hline 
$62\vdash66$ & 8 & 64 & 512\tabularnewline
\hline 
$66\vdash70$ & 5 & 68 & 340\tabularnewline
\hline 
$70\vdash74$ & 3 & 72 & 216\tabularnewline
\hline 
Total & 40 &  & 2440\tabularnewline
\hline 
\end{tabular} 
\par\end{center}

\begin{center}
Assim, $\overline{X}=\frac{2440}{40}=61$
\par\end{center}

\end{frame}

\begin{frame}{\foreignlanguage{english}{Média geométrica}}


\framesubtitle{Definição}
\selectlanguage{english}%
\begin{itemize}
\item \textbf{Média geométrica $(\overline{X_{g}})$:} é a raíz n-ésima
do produto dos dados, $\bar{X_{g}}=\sqrt[^{n}]{\prod_{i=1}^{n}X_{i}}$,
onde $\prod_{i=1}^{n}X_{i}=X_{1}.X_{2}....X_{n}$.

Ex: 10,60,360. $\overline{X_{g}}=\sqrt[3]{10.60.360}=60$.
\selectlanguage{brazil}%
\item Note que aplicando log, temos $log\overline{X_{g}}=\frac{1}{n}\sum_{i=1}^{n}logX_{i}$,
de modo que o logaritmo da média geométrica é igual à média aritmética
dos logaritmos dos valores observados.
\item Assim, podemos notar que a média geométrica é uma média aritmética
suavizada. Ela é muita utilizada em finanças e engenharia.
\end{itemize}
\end{frame}

\begin{frame}{\foreignlanguage{english}{Média geométrica}}


\framesubtitle{Relação com a média aritmética}
\begin{itemize}
\item Sempre teremos $\overline{X_{g}}\leq\overline{X}$, valendo a igualdade
apenas se $x_{i}=x_{j}$ para todo $i\neq j,$ ou seja, se todos os
dados são iguais.
\item Para provar que $\overline{X_{g}}\leq\overline{X}$ basta observar
que, para o caso de apenas duas observações, $X_{1}$ e $X_{2}$:

$\overline{X^{2}}-\overline{X_{g}^{2}}=(\frac{X_{1}+X_{2}}{2})^{2}-X_{1}.X_{2}$=\foreignlanguage{english}{\textrm{$\frac{X_{1}^{2}-2.X_{1}.X_{2}+X_{2}^{2}}{4}=(\frac{X_{1}-X_{2}}{2})\text{\texttwosuperior}\geq0$.}}

\selectlanguage{english}%
Logo, como $\overline{X^{2}}-\overline{X_{g}^{2}}\geq0$, temos $\overline{X_{g}}\leq\overline{X}$.
\end{itemize}
\end{frame}
\selectlanguage{brazil}%

\begin{frame}{Média geométrica ponderada}


\framesubtitle{Dados agrupados sem intervalo de classe}
\selectlanguage{english}%
\begin{itemize}
\item Se os dados forem agrupados, calculamos a \textit{\uline{média
geométrica ponderada}}, $\overline{X_{g}}=\sqrt[\overset{n}{\underset{i=1}{\sum}}fi]{\prod_{i=1}^{n}X_{i}^{f_{i}}}=\sqrt[\overset{n}{\underset{i=1}{\sum}}fi]{X_{1}^{f_{1}}.X_{2}^{f_{2}}...X_{n}^{f_{n}}}$.
Ex:
\end{itemize}
\begin{center}
\begin{tabular}{|c|c|}
\hline 
\selectlanguage{brazil}%
$X_{i}$\selectlanguage{english}%
 & \selectlanguage{brazil}%
$f_{i}$\selectlanguage{english}%
\tabularnewline
\hline 
\hline 
\selectlanguage{brazil}%
1\selectlanguage{english}%
 & \selectlanguage{brazil}%
2\selectlanguage{english}%
\tabularnewline
\hline 
\selectlanguage{brazil}%
3\selectlanguage{english}%
 & \selectlanguage{brazil}%
4\selectlanguage{english}%
\tabularnewline
\hline 
\selectlanguage{brazil}%
9\selectlanguage{english}%
 & \selectlanguage{brazil}%
2\selectlanguage{english}%
\tabularnewline
\hline 
\selectlanguage{brazil}%
27\selectlanguage{english}%
 & \selectlanguage{brazil}%
1\selectlanguage{english}%
\tabularnewline
\hline 
\selectlanguage{brazil}%
Total\selectlanguage{english}%
 & \selectlanguage{brazil}%
9\selectlanguage{english}%
\tabularnewline
\hline 
\end{tabular}
\par\end{center}

\begin{center}
$\overline{X_{g}}=\sqrt[9]{1^{2}.3^{4}.9^{2}.27^{1}}=3,8296$
\par\end{center}
\begin{itemize}
\item Se os dados forem agrupados com intervalo de classe o procedimento
é o mesmo, porém agora $X_{i}$ será o ponto médio de cada classe.
\end{itemize}
\end{frame}

\begin{frame}{\foreignlanguage{english}{Média harmônica}}


\framesubtitle{Definição}
\selectlanguage{english}%
\begin{itemize}
\item \textbf{Média harmônica $(\overline{X}_{h})$:} É o inverso da média
aritmética dos inversos de cada elemento do conjunto de dados, $\overline{X}_{h}=(\frac{1}{n}.\sum_{i=1}^{n}X_{i}^{-1})^{-1}=\frac{n}{\sum_{i=1}^{n}\frac{1}{X_{i}}}$.
\item A média harmônica é bastante utilizada na física, quando trabalhamos
com grandezas que variam inversamente. Ex: velocidade e tempo.
\item Sempre teremos \textrm{$\overline{X}_{h}\leq\overline{X}_{g}\leq\overline{X}$,
}valendo a igualdade apenas se todos os dados forem iguais. Podemos
ver isso para o caso de apenas duas observações, \foreignlanguage{brazil}{$X_{1}$
e $X_{2}$:}

\textrm{$\overline{X}_{h}=[\frac{1}{2}.(\frac{1}{X_{1}}+\frac{1}{X_{2}})]^{-1}=\frac{2X_{1}.X_{2}}{X_{1}+X_{2}}=\frac{2.\overline{X_{g}^{2}}}{2.\overline{X}}=\frac{\overline{X}_{g}.\overline{X}_{g}}{\overline{X}}$}.

Como $\overline{X}_{g}\leq\overline{X}$, temos $0\leq\frac{\overline{X_{g}}}{\overline{X}}\leq1.$
Logo, $\overline{X}_{h}\leq\overline{X}_{g}$.
\end{itemize}
\end{frame}
\selectlanguage{brazil}%

\begin{frame}{\foreignlanguage{english}{Média harmônica ponderada}}


\framesubtitle{Dados agrupados com intervalo de classe}
\selectlanguage{english}%
\begin{itemize}
\item Se os dados forem agrupados, calculamos a \textit{\uline{média
harmônica ponderada,}} $\overline{X}_{h}=\frac{\sum_{i=1}^{n}f_{i}}{\sum_{i=1}^{n}\frac{f_{i}}{X_{i}}}$.
Ex:
\end{itemize}
\begin{center}
\begin{tabular}{|c|c|c|c|}
\hline 
\selectlanguage{brazil}%
Classes\selectlanguage{english}%
 & \selectlanguage{brazil}%
$f_{i}$\selectlanguage{english}%
 & \selectlanguage{brazil}%
$X_{i}$\selectlanguage{english}%
 & \selectlanguage{brazil}%
$\frac{f_{i}}{X_{i}}$\selectlanguage{english}%
\tabularnewline
\hline 
\hline 
\selectlanguage{brazil}%
$1\vdash3$\selectlanguage{english}%
 & \selectlanguage{brazil}%
2\selectlanguage{english}%
 & \selectlanguage{brazil}%
2\selectlanguage{english}%
 & \selectlanguage{brazil}%
2/2=1\selectlanguage{english}%
\tabularnewline
\hline 
\textrm{$3\vdash5$} & \selectlanguage{brazil}%
4\selectlanguage{english}%
 & \selectlanguage{brazil}%
4\selectlanguage{english}%
 & \selectlanguage{brazil}%
4/4=1\selectlanguage{english}%
\tabularnewline
\hline 
\selectlanguage{brazil}%
$5\vdash7$\selectlanguage{english}%
 & \selectlanguage{brazil}%
8\selectlanguage{english}%
 & \selectlanguage{brazil}%
6\selectlanguage{english}%
 & \selectlanguage{brazil}%
8/6=1,33\selectlanguage{english}%
\tabularnewline
\hline 
\selectlanguage{brazil}%
$7\vdash9$\selectlanguage{english}%
 & \selectlanguage{brazil}%
4\selectlanguage{english}%
 & \selectlanguage{brazil}%
8\selectlanguage{english}%
 & \selectlanguage{brazil}%
4/8=0,5\selectlanguage{english}%
\tabularnewline
\hline 
\selectlanguage{brazil}%
$9\vdash11$\selectlanguage{english}%
 & \selectlanguage{brazil}%
2\selectlanguage{english}%
 & \selectlanguage{brazil}%
10\selectlanguage{english}%
 & \selectlanguage{brazil}%
2/10=0,2\selectlanguage{english}%
\tabularnewline
\hline 
\selectlanguage{brazil}%
total\selectlanguage{english}%
 & \selectlanguage{brazil}%
20\selectlanguage{english}%
 & \selectlanguage{brazil}%
\selectlanguage{english}%
 & \selectlanguage{brazil}%
4,03\selectlanguage{english}%
\tabularnewline
\hline 
\end{tabular}
\par\end{center}

\begin{center}
\textrm{$\overline{X_{h}}=\frac{20}{4,03}=4,96$}
\par\end{center}
\begin{itemize}
\item Se os dados forem agrupados sem intervalo de classe o procedimento
é o mesmo, porém agora $X_{i}$ será o valor de cada elemento do conjunto
de dados.
\end{itemize}
\end{frame}

\begin{frame}{\foreignlanguage{english}{Média harmônica}}


\framesubtitle{Observações}
\selectlanguage{english}%
\begin{itemize}
\item \textbf{Obs 1:} a média harmônica não aceita valores iguais a zero
como dados de uma série.
\selectlanguage{brazil}%
\item \textbf{Obs 2:} quando os valores da variável não forem muito diferentes,
verifica-se a seguinte relação,$\overline{X_{g}}=\frac{(\overline{X}+\overline{X_{h}})}{2}$.

Ex: \{10,1; 10,1; 10,2; 10,4; 10,5\}

$\overline{X}=\frac{51,3}{5}=10,2600$

$\overline{X_{h}}=\frac{5}{0,4874}=10,2574$

$\overline{X_{g}}=\frac{10,26+10,2574}{2}=10,2587$
\end{itemize}
\end{frame}

\begin{frame}{Moda}


\framesubtitle{Definição}
\begin{itemize}
\item \textbf{Moda $(M_{o})$:} É o valor que ocorre com maior frequência
em uma série de dados.
\item Se os dados estiverem não agrupados, devemos procurar o valor que
mais se repete.

Ex: \{7, 8, 9, 10, 10, 10, 11, 12\}. Temos Mo=10.
\item Se nenhum valor aparece mais vezes do que outro, chamamos a série
de amodal.

Ex: \{3, 5, 8, 10, 12\} não apresenta moda. É \textit{\uline{amodal}}.
\item Se dois ou mais valores repetem o mesmo número de vezes, a série tem
mais de um valor modal.

Ex: \{2, 3, 4, 4, 4, 5, 6, 7, 7, 7, 8, 9\} tem duas modas, é \textit{\uline{bimodal}}.
Suas modas são 4 e 7.
\item O emprego da moda é utilizado apenas em alguns casos específicos,
pois a média aritmética possui maior estabilidade. Ex: salários.
\end{itemize}
\end{frame}

\begin{frame}{Moda}


\framesubtitle{Dados agrupados}
\begin{itemize}
\item \textbf{Sem intervalos de classe:} com os dados agrupados, é possível
determinar a moda apenas olhando o dado com a maior frequência.
\end{itemize}
\begin{center}
\begin{tabular}{|c|c|}
\hline 
Temperatura & Frequência\tabularnewline
\hline 
\hline 
0ºC & 3\tabularnewline
\hline 
1ºC & 9\tabularnewline
\hline 
2ºC & 12\tabularnewline
\hline 
3ºC & 6\tabularnewline
\hline 
\end{tabular} 
\par\end{center}

\begin{center}
$M_{o}=2\text{ºC}$
\par\end{center}
\begin{itemize}
\item \textbf{Com intervalos de classe:} a classe com a maior frequência
é a classe modal. A moda será um valor compreendido entre os limites
da classe moda.
\end{itemize}
\end{frame}

\begin{frame}{Moda}


\framesubtitle{Dados agrupados}
\begin{itemize}
\item \textbf{Moda bruta:} $M_{o}=(\frac{l*+L*}{2}),$ onde \foreignlanguage{english}{\textrm{$l*$}}
é o limite inferior da classe modal e $L*$ o limite superior da classe
modal.
\item \textbf{Moda de Czuber: }$M_{c}=l*+(\frac{d_{1}}{d_{1}+d_{2}}).h*,$
onde $d_{1}$ é a frequência da classe modal menos a frequência da
classe anterior a modal; $d_{2}$ é a frequência da classe modal menos
a frequência da classe posterior a modal; e $h*$ é a amplitude da
classe modal. Ex:
\end{itemize}
\begin{center}
\begin{tabular}{|c|c|}
\hline 
Classes & Frequências\tabularnewline
\hline 
\hline 
$54\vdash58$ & 9\tabularnewline
\hline 
$58\vdash62$ & 11\tabularnewline
\hline 
$62\vdash66$ & 8\tabularnewline
\hline 
$66\vdash70$ & 5\tabularnewline
\hline 
\end{tabular} 
\par\end{center}

\begin{center}
A classe modal é $58\vdash62$, logo $M_{o}=\frac{58+62}{2}=60;$
e $M_{c}=58+\frac{(11-9)}{(11-9)+(11-8)}.4=59,6$
\par\end{center}

\end{frame}

\begin{frame}{Mediana}


\framesubtitle{Definição}
\begin{itemize}
\item \textbf{Mediana $(M_{e})$:} é o valor que separa um conjunto de dados
(dispostos em ordem crescente ou decrescente) em dois subconjuntos
de mesmo número de elementos.
\item Se a série tiver número ímpar de termos, a mediana será o elemento
$\frac{n+1}{2}.$
\item Se a série tiver numero par de termos, a mediana será a média dos
elementos $\frac{n}{2}$ e $\frac{n}{2}+1.$
\item Ex. 1: \{1, 3, 0, 0, 2, 4, 1, 2, 5\}.

1º colocamos a série em ordem crescente \{0, 0, 1, 1, 2, 2, 3, 4,
5\}.

2º como existem 9 elementos, a mediana será o elemento de número $\frac{n+1}{2}=\frac{10}{2}=5$.
Assim, $M_{e}=2$.
\end{itemize}
\end{frame}

\begin{frame}{Mediana}


\framesubtitle{Definição}
\begin{itemize}
\item Ex. 2: \{1, 3, 0, 0, 2, 4, 1, 3, 5, 6\}.

1º colocamos a série em ordem crescente \{0, 0, 1, 1, 2, 3, 3, 4,
5, 6\}.

2º como existem dez elementos, a mediana será a média dos elementos
de número $\frac{n}{2}=\frac{10}{2}=5$ e $\frac{n}{2}+1=6,$ logo
$M_{e}=\frac{2+3}{2}=2,5$.
\item A mediana depende da posição dos valores da série. Ela não se deixa
influenciar por valores extremos, como é o caso da média. Já a moda,
depende da frequência. Estes três valores em geral são diferentes.
\item Ex. 3: \{5, 7, 10, 10, 18\}. $\overline{X}=10$, $M_{o}=10$, \foreignlanguage{english}{\textrm{$M_{e}=10$}}.
\item Ex. 4: \{5, 5, 10, 13, 67\}. $\overline{X}=20$, $M_{o}=5$, $M_{e}=10$.
\end{itemize}
\end{frame}

\begin{frame}{Mediana}


\framesubtitle{Dados agrupados sem intervalo de classe}
\begin{itemize}
\item Se o somatório das frequências for ímpar, a mediana será o elemento
$\frac{\sum_{i=1}^{n}fi+1}{2}$.
\item Identificamos facilmente esse elemento através da frequência acumulada.
Ex:
\end{itemize}
\begin{center}
\begin{tabular}{|c|c|c|}
\hline 
Xi & fi & Fi\tabularnewline
\hline 
\hline 
0 & 2 & 2\tabularnewline
\hline 
1 & 6 & 8\tabularnewline
\hline 
2 & 9 & 17\tabularnewline
\hline 
3 & 13 & 30\tabularnewline
\hline 
4 & 5 & 35\tabularnewline
\hline 
total & 35 & \tabularnewline
\hline 
\end{tabular} 
\par\end{center}

\begin{center}
Temos $\frac{\sum fi+1}{2}=\frac{36}{2}=18$. Logo, $M_{e}=3$.
\par\end{center}

\end{frame}

\begin{frame}{Mediana}


\framesubtitle{Dados agrupados sem intervalo de classe}
\begin{itemize}
\item Se o somatório das frequências for par, a mediana será a média dos
termos $\frac{\sum_{i=1}^{n}f_{i}}{2}$ e $\frac{\sum_{i=1}^{n}f_{i}}{2}+1$.
Ex:
\end{itemize}
\begin{center}
\begin{tabular}{|c|c|c|}
\hline 
Xi & fi & Fi\tabularnewline
\hline 
\hline 
12 & 1 & 1\tabularnewline
\hline 
14 & 2 & 3\tabularnewline
\hline 
15 & 1 & 4\tabularnewline
\hline 
16 & 2 & 6\tabularnewline
\hline 
17 & 1 & 7\tabularnewline
\hline 
20 & 1 & 8\tabularnewline
\hline 
total & 8 & \tabularnewline
\hline 
\end{tabular} 
\par\end{center}

\begin{center}
Temos $\frac{\sum_{i=1}^{n}f_{i}}{2}=\frac{8}{2}=4$ e $\frac{\sum_{i=1}^{n}f_{i}}{2}+1=\frac{8}{2}+1=5$.
Assim, $M_{e}=\frac{15+16}{2}=15,5$.
\par\end{center}

\end{frame}

\begin{frame}{Mediana}


\framesubtitle{Dados agrupados com intervalo de classe}
\begin{itemize}
\item Para calcularmos a mediana de dados agrupados com intervalo de classe,
seguimos as seguintes etapas:

1º Determinamos as frequências acumuladas.

2º Calculamos $\frac{\overset{n}{\underset{i=1}{\sum}}fi}{2}$.

3º Marcamos a classe correspondente a frequência acumulada imediatamente
superior a $\frac{\overset{n}{\underset{i=1}{\sum}}fi}{2}$. Essa
será a classe mediana.

4º Temos que$M_{e}=l^{*}+\frac{[(\frac{\sum_{i=1}^{n}f_{i}}{2}-FAA).h*]}{f*}$,
onde $l*$ é o limite inferior da classe mediana, $FAA$ é a frequência
acumulada da classe anterior à classe mediana, $f*$ é a frequência
simples da classe mediana, e $h*$ é a amplitude do intervalo da classe
mediana.
\end{itemize}
\end{frame}

\begin{frame}{Mediana}


\framesubtitle{Dados agrupados com intervalo de classe}
\begin{itemize}
\item Ex:
\end{itemize}
\begin{center}
\begin{tabular}{|c|c|c|}
\hline 
Classes & fi & Fi\tabularnewline
\hline 
\hline 
$50\vdash54$ & 4 & 4\tabularnewline
\hline 
\selectlanguage{english}%
\textrm{$54\vdash58$}\selectlanguage{brazil}%
 & 9 & 13\tabularnewline
\hline 
$58\vdash62$ & 11 & 24\tabularnewline
\hline 
$62\vdash66$ & 8 & 32\tabularnewline
\hline 
$66\vdash70$ & 5 & 37\tabularnewline
\hline 
$70\vdash74$ & 3 & 40\tabularnewline
\hline 
total & 40 & \tabularnewline
\hline 
\end{tabular}
\par\end{center}

\selectlanguage{english}%
\begin{center}
\textrm{$\frac{\sum fi}{2}=20$ $\rightarrow$ Classe mediana: $58\vdash62$.}
\par\end{center}

\begin{center}
$l*=58$, $FAA=13$, $f*=11$, $h*=4$.
\par\end{center}

\begin{center}
$Md=58+\frac{[(20-13)x4]}{11}=58+\frac{28}{11}=60,54$.
\par\end{center}

\end{frame}
\selectlanguage{brazil}%

\begin{frame}{\foreignlanguage{english}{Mediana}}


\framesubtitle{Possíveis empregos da mediana}
\selectlanguage{english}%
\begin{itemize}
\item Quando desejamos obter o ponto que divide a distribuição em duas partes
iguais.
\item Quando há valores extremos que afetam demais a média.
\item Em variáveis como salário.
\end{itemize}
\end{frame}

\begin{frame}{Separatrizes}


\framesubtitle{Definição}
\begin{itemize}
\item Existem outras medidas de posição que não são medidas de tendência
central, como os quartis, decis e percentis, conhecidas genericamente
por separatrizes.
\item \textbf{Quartil:} são os valores que dividem a série em quatro partes
iguais. Precisamos 3 quartis para dividir a série em quatro partes.
\item Note que o segundo quartil $(Q_{2})$ será sempre igual a mediana.
\end{itemize}
\end{frame}

\begin{frame}{Quartil}


\framesubtitle{Dados não agrupados}
\begin{itemize}
\item Ex. 1: \{5, 2, 6, 9, 10, 13, 15\}.

1º ordenamos a série \{2, 5, 6, 9, 10, 13, 15\}.

2º calculamos a mediana, que será o segundo quartil $M_{e}=Q2=9$.

3º dividimos a série em dois grupos \{2, 5, 6\} e \{10, 13, 15\}.

4º calculamos os outros quartis como sendo as medianas desses dois
grupos $Q_{1}=5$ e $Q_{2}=13$.
\item Ex. 2: \{1, 1, 2, 3, 5, 5, 6, 7, 9, 9, 10, 13\}.

$M_{e}=Q_{2}=\frac{5+6}{2}=5,5$. Logo, temos \{1, 1, 2, 3, 5, 5\}
com $Q_{1}=2,5$, e \{6, 7, 9, 9, 10, 13\} com $Q_{3}=9$.
\end{itemize}
\end{frame}

\begin{frame}{Quartil}


\framesubtitle{Dados agrupados}
\begin{itemize}
\item Se os dados forem agrupados sem intervalos de classe, utilizamos $\frac{\sum_{i=1}^{n}f_{i}}{2}$
e $\frac{\sum_{i=1}^{n}f_{i}}{2}+1$ para calcular as posições dos
quartis.
\item Se os dados forem agrupados com intervalos de classe, utilizamos a
mesma fórmula da mediana para calcular os quartis, entretanto substituímos
$\frac{\sum_{i=1}^{n}f_{i}}{2}$ por $k\frac{\sum_{i=1}^{n}f_{i}}{4},$
sendo k o número do quartil:

Q1=$l^{*}+\frac{[(\frac{\sum fi}{4}-FAA).h^{*}]}{f^{*}}$

Q2= $l^{*}+\frac{[(2\frac{\sum fi}{4}-FAA).h^{*}]}{f^{*}}$

Q3=$l^{*}+\frac{[(3\frac{\sum fi}{4}-FAA).h^{*}]}{f^{*}}$
\end{itemize}
\end{frame}

\begin{frame}{Quartil}


\framesubtitle{Dados agrupados com intervalo de classe}
\begin{itemize}
\item Ex:
\end{itemize}
\begin{center}
\begin{tabular}{|c|c|c|}
\hline 
Classes & fi & Fi\tabularnewline
\hline 
\hline 
$50\vdash54$ & 4 & 4\tabularnewline
\hline 
$54\vdash58$ & 9 & 13\tabularnewline
\hline 
$58\vdash62$ & 11 & 24\tabularnewline
\hline 
$62\vdash66$ & 8 & 32\tabularnewline
\hline 
$66\vdash70$ & 5 & 37\tabularnewline
\hline 
$70\vdash74$ & 3 & 40\tabularnewline
\hline 
total & 40 & \tabularnewline
\hline 
\end{tabular}
\par\end{center}

\begin{center}
$\frac{\sum fi}{2}=20$ \foreignlanguage{english}{$\rightarrow$ {\footnotesize{}classe
mediana}: $58\vdash62.$ $l*=58$, $FAA=13$, $f*=11$, $h*=4$}
\par\end{center}

\begin{center}
$M_{e}=Q_{2}=58+\frac{[(20-13)x4)}{11}=60,54$.
\par\end{center}

\begin{center}
$\frac{\sum fi}{4}=10$ $\rightarrow$ {\footnotesize{}classe mediana
do 1º grupo}: $54\vdash58$. $Q1=54+\frac{[(10-4)x4]}{9}=56,66$.
\par\end{center}

\begin{center}
$\frac{3.\sum fi}{4}=30$ $\rightarrow${\footnotesize{}classe mediana
do 3º grupo}: $62\vdash66$. $Q3=62+\frac{[(30-24)x4]}{8}=65$.
\par\end{center}

\end{frame}

\begin{frame}{Decil}


\framesubtitle{Definição}
\begin{itemize}
\item \textbf{Decil:} são os valores que dividem a série em dez partes iguais.
Precisamos 9 decis para dividir a série em 10 partes.
\item O procedimento é análogo, porém agora o 5º decil será igual ao 2º
quartil, que será igual à mediana.
\item Ex: calcule o 3º decil da tabela anterior.

Como k=3, temos $3.\frac{\sum fi}{10}=3.\frac{40}{10}=12,$ e a classe
mediana é $54\vdash58$. Logo, $D_{3}=54+\frac{[(12-4)x4]}{9}=57,55$.
\end{itemize}
\end{frame}

\begin{frame}{Percentil (ou centil)}

\begin{itemize}
\item Serão os 99 valores que separam a série em 100 partes iguais, de modo
que $P_{50}=M_{e}$, $P_{25}=Q_{1}$ e $P_{75}=Q_{3}$. O cálculo
é análogo, mas utilizando$k.\frac{\sum fi}{100}$.
\end{itemize}
\end{frame}

\end{document}
