\documentclass[12pt,ignorenonframetext,,aspectratio=149]{beamer}
\setbeamertemplate{caption}[numbered]
\setbeamertemplate{caption label separator}{: }
\setbeamercolor{caption name}{fg=normal text.fg}
\usepackage{lmodern}
\usepackage{amssymb,amsmath}
\usepackage{ifxetex,ifluatex}
\usepackage{fixltx2e} % provides \textsubscript
\ifnum 0\ifxetex 1\fi\ifluatex 1\fi=0 % if pdftex
  \usepackage[T1]{fontenc}
  \usepackage[utf8]{inputenc}
\else % if luatex or xelatex
  \ifxetex
    \usepackage{mathspec}
  \else
    \usepackage{fontspec}
  \fi
  \defaultfontfeatures{Ligatures=TeX,Scale=MatchLowercase}
  \newcommand{\euro}{€}
\fi
% use upquote if available, for straight quotes in verbatim environments
\IfFileExists{upquote.sty}{\usepackage{upquote}}{}
% use microtype if available
\IfFileExists{microtype.sty}{%
\usepackage{microtype}
\UseMicrotypeSet[protrusion]{basicmath} % disable protrusion for tt fonts
}{}

% Comment these out if you don't want a slide with just the
% part/section/subsection/subsubsection title:
\AtBeginPart{
  \let\insertpartnumber\relax
  \let\partname\relax
  \frame{\partpage}
}
\AtBeginSection{
  \let\insertsectionnumber\relax
  \let\sectionname\relax
  \frame{\sectionpage}
}
\AtBeginSubsection{
  \let\insertsubsectionnumber\relax
  \let\subsectionname\relax
  \frame{\subsectionpage}
}

\setlength{\emergencystretch}{3em}  % prevent overfull lines
\providecommand{\tightlist}{%
  \setlength{\itemsep}{0pt}\setlength{\parskip}{0pt}}
\setcounter{secnumdepth}{0}

\title{Métodos Estatísticos Básicos}
\subtitle{Aula 3 - Medidas de Tendência Central e de Posição}
\author{Regis A. Ely}
\date{16 de maio de 2020}

%% Here's everything I added.
%%--------------------------

\usepackage{graphicx}
\usepackage{rotating}
%\setbeamertemplate{caption}[numbered]
\usepackage{hyperref}
\usepackage{caption}
\usepackage[normalem]{ulem}
%\mode<presentation>
\usepackage{wasysym}
%\usepackage{amsmath}


% Get rid of navigation symbols.
%-------------------------------
\setbeamertemplate{navigation symbols}{}

% Optional institute tags and titlegraphic.
% Do feel free to change the titlegraphic if you don't want it as a Markdown field.
%----------------------------------------------------------------------------------
\institute{Departamento de Economia\\
Universidade Federal de Pelotas}

% \titlegraphic{\includegraphics[width=0.3\paperwidth]{\string~/Dropbox/teaching/clemson-academic.png}} % <-- if you want to know what this looks like without it as a Markdown field. 
% -----------------------------------------------------------------------------------------------------


% Some additional title page adjustments.
%----------------------------------------
\setbeamertemplate{title page}[empty]
%\date{}
\setbeamerfont{subtitle}{size=\small}

\setbeamercovered{transparent}

% Some optional colors. Change or add as you see fit.
%---------------------------------------------------
\definecolor{clemsonpurple}{HTML}{522D80}
 \definecolor{clemsonorange}{HTML}{F66733}
\definecolor{uiucblue}{HTML}{003C7D}
\definecolor{uiucorange}{HTML}{F47F24}


% Some optional color adjustments to Beamer. Change as you see fit.
%------------------------------------------------------------------
\setbeamercolor{frametitle}{fg=clemsonpurple,bg=white}
\setbeamercolor{title}{fg=clemsonpurple,bg=white}
\setbeamercolor{local structure}{fg=clemsonpurple}
\setbeamercolor{section in toc}{fg=clemsonpurple,bg=white}
% \setbeamercolor{subsection in toc}{fg=clemsonorange,bg=white}
\setbeamercolor{footline}{fg=clemsonpurple!50, bg=white}
\setbeamercolor{block title}{fg=clemsonorange,bg=white}


\let\Tiny=\tiny


% Sections and subsections should not get their own damn slide.
%--------------------------------------------------------------
\AtBeginPart{}
\AtBeginSection{}
\AtBeginSubsection{}
\AtBeginSubsubsection{}

% Suppress some of Markdown's weird default vertical spacing.
%------------------------------------------------------------
\setlength{\emergencystretch}{0em}  % prevent overfull lines
\setlength{\parskip}{0pt}


% Allow for those simple two-tone footlines I like. 
% Edit the colors as you see fit.
%--------------------------------------------------
\defbeamertemplate*{footline}{my footline}{%
    \ifnum\insertpagenumber=1
    \hbox{%
        \begin{beamercolorbox}[wd=\paperwidth,ht=.8ex,dp=1ex,center]{}%
      % empty environment to raise height
        \end{beamercolorbox}%
    }%
    \vskip0pt%
    \else%
        \Tiny{%
            \hfill%
		\vspace*{1pt}%
            \insertframenumber/\inserttotalframenumber \hspace*{0.1cm}%
            \newline%
            \color{clemsonpurple}{\rule{\paperwidth}{0.4mm}}\newline%
            \color{clemsonorange}{\rule{\paperwidth}{.4mm}}%
        }%
    \fi%
}

% Various cosmetic things, though I must confess I forget what exactly these do and why I included them.
%-------------------------------------------------------------------------------------------------------
\setbeamercolor{structure}{fg=blue}
\setbeamercolor{local structure}{parent=structure}
\setbeamercolor{item projected}{parent=item,use=item,fg=clemsonpurple,bg=white}
\setbeamercolor{enumerate item}{parent=item}

% Adjust some item elements. More cosmetic things.
%-------------------------------------------------
\setbeamertemplate{itemize item}{\color{clemsonpurple}$\bullet$}
\setbeamertemplate{itemize subitem}{\color{clemsonpurple}\scriptsize{$\bullet$}}
\setbeamertemplate{itemize/enumerate body end}{\vspace{.6\baselineskip}} % So I'm less inclined to use \medskip and \bigskip in Markdown.

% Automatically center images
% ---------------------------
% Note: this is for ![](image.png) images
% Use "fig.align = "center" for R chunks

\usepackage{etoolbox}

\AtBeginDocument{%
  \letcs\oig{@orig\string\includegraphics}%
  \renewcommand<>\includegraphics[2][]{%
    \only#3{%
      {\centering\oig[{#1}]{#2}\par}%
    }%
  }%
}

% I think I've moved to xelatex now. Here's some stuff for that.
% --------------------------------------------------------------
% I could customize/generalize this more but the truth is it works for my circumstances.

\ifxetex
\setbeamerfont{title}{family=\fontspec{serif}}
\setbeamerfont{frametitle}{family=\fontspec{serif}}
\usepackage[font=small,skip=0pt]{caption}
 \else
 \fi

% Okay, and begin the actual document...

\begin{document}
\frame{\titlepage}

\section[]{}
\frame{\small \frametitle{Conteúdo}
\tableofcontents}

\hypertarget{medidas-de-tenduxeancia-central}{%
\section{Medidas de tendência
central}\label{medidas-de-tenduxeancia-central}}

\begin{frame}{Medidas de tendência central}
\protect\hypertarget{medidas-de-tenduxeancia-central-1}{}
Medidas de tendência central são estatísticas que representam o ponto
central de um conjunto de dados

\begin{itemize}
\tightlist
\item
  \textbf{Média}: os tipos mais comuns são a média \emph{aritmética},
  \emph{harmônica} e \emph{geométrica}
\item
  \textbf{Mediana} valor que separa a metade maior e a metade menor de
  um conjunto de dados
\item
  \textbf{Moda}: valor mais frequente de um conjunto de dados
\end{itemize}
\end{frame}

\hypertarget{muxe9dia-aritmuxe9tica}{%
\subsection{Média aritmética}\label{muxe9dia-aritmuxe9tica}}

\begin{frame}{Média aritmética simples}
\protect\hypertarget{muxe9dia-aritmuxe9tica-simples}{}
\textbf{Média aritmética}: razão entre a soma dos valores e o número de
observações de um conjunto de dados
\(\bar{X}=\frac{{\sum_{i=1}^{n}X_{i}}}{n}\)

Quando temos dados não-agrupados pela frequência das observações,
calculamos a média aritmética simples

\begin{theorem*}
Ex: 10, 14, 13, 15, 16, 18, 12
\end{theorem*}

Logo, \(\overline{X}=\frac{(10+14+13+15+16+18+12)}{7}=14\).

Desvio em relação à média:\} é a diferença entre um elemento de um
conjunto de valores e a média aritmética desse conjunto, ou seja,
\(d_{i}=X_{i}-\overline{X}.\)

Ex: \(d_{1}=10-14=-4\); \(d_{2}=14-14=0\), etc.
\end{frame}

\begin{frame}{Média aritmética ponderada}
\protect\hypertarget{muxe9dia-aritmuxe9tica-ponderada}{}
\end{frame}

\begin{frame}{Propriedades da média aritmética}
\protect\hypertarget{propriedades-da-muxe9dia-aritmuxe9tica}{}
\end{frame}

\hypertarget{muxe9dia-geomuxe9trica}{%
\subsection{Média geométrica}\label{muxe9dia-geomuxe9trica}}

\begin{frame}{Média geométrica simples}
\protect\hypertarget{muxe9dia-geomuxe9trica-simples}{}
\end{frame}

\begin{frame}{Média geométrica ponderada}
\protect\hypertarget{muxe9dia-geomuxe9trica-ponderada}{}
\end{frame}

\begin{frame}{Propriedades da média geométrica}
\protect\hypertarget{propriedades-da-muxe9dia-geomuxe9trica}{}
\end{frame}

\hypertarget{muxe9dia-harmuxf4nica}{%
\subsection{Média harmônica}\label{muxe9dia-harmuxf4nica}}

\begin{frame}{Média harmônica simples}
\protect\hypertarget{muxe9dia-harmuxf4nica-simples}{}
\end{frame}

\begin{frame}{Média harmônica ponderada}
\protect\hypertarget{muxe9dia-harmuxf4nica-ponderada}{}
\end{frame}

\begin{frame}{Propriedades da média harmônica}
\protect\hypertarget{propriedades-da-muxe9dia-harmuxf4nica}{}
\end{frame}

\hypertarget{moda}{%
\subsection{Moda}\label{moda}}

\begin{frame}{Moda simples}
\protect\hypertarget{moda-simples}{}
\end{frame}

\begin{frame}{Moda com intervalos de classe}
\protect\hypertarget{moda-com-intervalos-de-classe}{}
\end{frame}

\hypertarget{mediana}{%
\subsection{Mediana}\label{mediana}}

\begin{frame}{Mediana simples}
\protect\hypertarget{mediana-simples}{}
\end{frame}

\begin{frame}{Mediana com intervalos de classe}
\protect\hypertarget{mediana-com-intervalos-de-classe}{}
\end{frame}

\hypertarget{medidas-de-posiuxe7uxe3o}{%
\section{Medidas de posição}\label{medidas-de-posiuxe7uxe3o}}

\begin{frame}{Separatrizes}
\protect\hypertarget{separatrizes}{}
Medidas de posição são estatísticas que representam a tendência de
concentração de um conjunto de dados ao redor de certos
pontos\hfill\break

As medidas de posição mais usuais são chamadas \emph{separatrizes} e
incluem:

\begin{itemize}
\tightlist
\item
  \textbf{Quartil}
\item
  \textbf{Decil}
\item
  \textbf{Percentil}
\end{itemize}
\end{frame}

\hypertarget{quartil}{%
\subsection{Quartil}\label{quartil}}

\begin{frame}{Quartil simples}
\protect\hypertarget{quartil-simples}{}
\end{frame}

\begin{frame}{Quartil com intervalos de classe}
\protect\hypertarget{quartil-com-intervalos-de-classe}{}
\end{frame}

\hypertarget{decil}{%
\subsection{Decil}\label{decil}}

\begin{frame}{Decil simples}
\protect\hypertarget{decil-simples}{}
\end{frame}

\begin{frame}{Decil com intervalos de classe}
\protect\hypertarget{decil-com-intervalos-de-classe}{}
\end{frame}

\hypertarget{percentil}{%
\subsection{Percentil}\label{percentil}}

\begin{frame}{Decil simples}
\protect\hypertarget{decil-simples-1}{}
\end{frame}

\begin{frame}{Decil com intervalos de classe}
\protect\hypertarget{decil-com-intervalos-de-classe-1}{}
\end{frame}

\hypertarget{exemplo-no-r}{%
\section{Exemplo no R}\label{exemplo-no-r}}

\begin{frame}{Exemplo no R}
\protect\hypertarget{exemplo-no-r-1}{}
\end{frame}


\end{document}